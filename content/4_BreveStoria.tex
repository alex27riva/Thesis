\mcchap{La visualizzazione dati}{cap:breveStoria}

\section{Cos'è la visualizzazione dati}
La rappresentazione grafica dei dati o in inglese \emph{data visualization} è l’attività di comunicazione realizzata tramite la proiezione di dati in una forma grafica strutturata.

Mediante l’utilizzo di elementi visivi come grafici o mappe, si ha una lettura dei dati che evidenzia le informazioni rilevanti e consente di riconoscere eventuali valori anomali.
Nel mondo dei big data, gli strumenti e le tecnologie di data visualization sono essenziali per analizzare enormi quantità di informazioni e prendere decisioni data oriented.

\section{Breve storia}

Con la tecnologia moderna, la visualizzazione di dati non è mai stata così semplice.
Con pochi click si può trasformare un enorme tabella di dati grezzi in un diagramma accattivante e semplice da interpretare.
Rappresentare i dati sotto forma di grafici, fornisce una via molto rapida per rendere comprensibile e far capire il proprio punto di vista.
Molte tecniche di visualizzazione dati che si usano oggi sono state inventate durante la prima rivoluzione industriale.
Quello che può sembrare semplice e ovvio al giorno d’oggi, come un grafico a linea o a barre, sarebbe stato strano e incomprensibile duecento anni fa.

Nel 18esimo secolo uno tra i primi pionieri della visualizzazione dati fu William Playfair, uno statista scozzese, visse nel periodo dell’Illuminismo e che introdusse la rappresentazione grafica in statistica ed economia.

In pochi anni introdusse modi innovativi di rappresentare graficamente i dati, oggi passati alla storia. Nella sua opera del 1786 dal titolo \emph{The Commercial and Political Atlas} incluse 44 grafici, tutti particolarmente \emph{creativi} per l’epoca, come dimostra quello che illustra la differenza tra import ed export dell’economia inglese, nella figura \ref{fig:atlante_politico_commerciale}.

Ai tempi di Playfair, i dati venivano presentati in tabelle asettiche di non facile lettura con poca attenzione per la loro interpretazione. Per comprenderli, non c’erano scorciatoie intuitive, ma un laborioso lavoro di esaminare attentamente i numeri, ricordare, copiare e confrontare i dati.

La figura \ref{fig:diagramma_temporale} mostra il diagramma inventato da Joseph Priestly dove con una linea temporale, rende immediatamente visibile quali personaggi storici fossero contemporanei.

Playfair fu anche l’inventore del grafico a torta pubblicato per la prima volta nel 1801 nel \emph{Statistical Breviary} per descrivere la distribuzione dell’Impero Ottomano tra i vari continenti europeo, africano e asiatico (nella figura \ref{fig:playfair_grafico_torta}).

\begin{figure}[htp]
    \centering
    \includegraphics[width=10cm]{PriestleyChart}
    \caption{Joseph Priestley — Diagramma temporale}
    \label{fig:diagramma_temporale}
\end{figure}

\begin{figure}[htp]
    \centering
    \includegraphics[width=9cm]{Playfair_chart}
    \caption{William Playfair - Atlante Politico e Commerciale}
    \label{fig:atlante_politico_commerciale}
\end{figure}

\begin{figure}[htp]
    \centering
    \includegraphics[width=6cm]{playfair_piechart}
    \caption{William Playfair — Grafico a torta}
    \label{fig:playfair_grafico_torta}
\end{figure}