\mcchap{La visualizzazione dati}{cap:breveStoria}

\section{Cos'è la visualizzazione dati }
La rappresentazione grafica dei dati o in inglese \emph{data visualization} è l’attività di comunicazione realizzata tramite la proiezione di dati in una forma grafica strutturata.

\section{Breve storia}

Con la tecnologia moderna, la visualizzazione di dati non è mai stata così semplice.
Con pochi click si può trasformare un enorme tabella di dati grezzi in un diagramma accattivante e semplice da interpretare.
Rappresentare i dati sotto forma di grafici, fornisce una via molto rapida per rendere comprensibile e far capire il proprio punto di vista.
Molte tecniche di visualizzazione dati che si usano oggi sono state inventate durante la prima rivoluzione industriale.
Quello che può sembrare semplice e ovvio al giorno  oggi, come un grafico a linea o a barre,  sarebbe stato strano e incomprensibile  200 anni fa.

\noindent Nel 18esimo secolo uno tra i primi pionieri della visualizzazione dati  fu William Playfair, che tradusse dati in grafici. 

\noindent William Playfair, nella sua tortuosa carriera, oggi impensabile, si dedicò: allo spionaggio, all’ingegneria, alla redazione, alla contabilità, all’invenzione, alla lavorazione dei metalli e molto altro. La sua eredità duratura risiede nel campo della statistica, con i grafici che ha progettato che costituiscono il nucleo della visualizzazione dei dati di oggi.

\noindent Ai tempi di Playfair, i dati venivano presentati in tabelle asettiche di non facile lettura con poca attenzione per la loro interpretazione. Per comprenderli, non c’erano scorciatoie intuitive, dove solo dopo un laborioso lavoro di esaminare attentamente i numeri, ricordare, copiare e confrontare i dati.

\noindent Poi, nel 1765, arrivò il diagramma temporale di Joseph Priestley, che mostrava le vite sovrapposte di vari statisti e filosofi classici. Invece di elencare semplicemente nomi, anni di nascita e anni di morte, Priestley li tracciò su una linea temporale, rendendo immediatamente visibile quali personaggi storici fossero contemporanei.

\begin{figure}[htp]
    \centering
    \includegraphics[width=10cm]{PriestleyChart}
    \caption{Joseph Priestley — Grafico sequenza temporale}
    \label{fig:PriestleyChart}
\end{figure}

\noindent Queste sequenze temporali si sono rivelate un successo e hanno ispirato direttamente l'invenzione del grafico a barre di Playfair, che apparve per la prima volta nel suo Atlante commerciale e politico.

\begin{figure}[htp]
    \centering
    \includegraphics[width=9cm]{Playfair_chart}
    \caption{William Playfair — Atlante Politico e Commerciale}
    \label{fig:playfair_chart}
\end{figure}

\noindent Quindici anni dopo, Playfair era tornato con con il grafico a torta a volte controverso e varie combinazioni creative. È sorprendente pensare che, più di duecento anni dopo, le idee di un uomo costituiscano ancora la maggior parte delle opzioni grafiche in un software di visualizzazione dei dati all'avanguardia.

\begin{figure}[htp]
    \centering
    \includegraphics[width=6cm]{playfair_piechart}
    \caption{William Playfair — Grafico a torta}
    \label{fig:playfair_piechart}
\end{figure}




