\mcchap{Contesto del lavoro}{cap:contesto}

Il 30 dicembre 2019, Li Wenliang, un oftalmologista dell’ospedale di Wuhan, ha condiviso online in una chat dei suoi amici la notizia di un nuovo virus sospetto simile alla SARS.
Il 3 gennaio 2020, Li fu intimato dalla polizia di non diffondere false notizie online.

Li fu infettato dal nuovo virus e morì poco dopo, il 7 febbraio all’età di 33 anni.
Il virus scoperto in Cina, si diffuse rapidamente, fino a diventare un'emergenza sanitaria a livello globale.

A causa dell’elevata contagiosità del virus, e alla necessità di cure in terapia intensiva per il numero elevato di ricoveri, gli ospedali andarono sotto pressione e i vari governi del mondo misero in atto diverse misure restrittive per contenere il propagarsi del virus, quando l'epidemia si trasformò in pandemia.

Il seguente lavoro di tesi, in collaborazione con il Professor Davide Tosi Ricercatore dell’Università degli Studi dell’Insubria di Varese, autore di diversi articoli scientifici\cite{tosi_1}\cite{tosi_2}\cite{tosi_3}, ha lo scopo di illustrare mediante dei grafici interattivi l’evoluzione in tempo reale dei dati dell’epidemia in Italia, e rappresenta il prosieguo del lavoro di tesi intrapreso dalla Dottoressa Alice Schiavone sul calcolo degli indici di contagio R0(t)\cite{schiavone_tesi}.