\mcchap{Contesto del lavoro}{cap:contesto}
\setlength{\parskip}{1em}
\section{Origine}
Il 30 dicembre 2019, Li Wenliang, un oculista dell’ospedale di Wuhan, ha condiviso online in una chat dei suoi amici la notizia di un nuovo virus sospetto simile alla SARS.
Il 3 gennaio 2020, la polizia disse a Li di smetterla di diffondere false notizie online.

\noindent Li torno al lavoro e fu infettato dal nuovo virus.
Morì poco dopo, il 7 febbraio al età di 33 anni.
Il virus originato in Cina, rapidamente si diffuse, fino a diventare un'emergenza sanitaria globale. Gli ospedali divennero presto sovraffollati a causa dell’elevata contagiosità del virus, e le persone malate richiedono speciali cure in terapia intensiva.

\noindent I vari governi del mondo misero in atto diverse misure per cercare di contenere il virus, ma nessuno ci riuscì, e il da epidemia si trasformò in una pandemia globale.


\noindent Il seguente lavoro di tesi, in collaborazione con il Prof. Davide Tosi,  è la continuazione di quello iniziato da Alice Schiavone sul calcolo degli indici di contagio R0(t).