\mcchap{Introduzione}{cap:intro}
\setlength{\parskip}{1em}


Viviamo in un mondo sempre più data-driven. La quantità di dati prodotti ogni giorno cresce in modo esponenziale.
%I dati sono fondamentali per prendere decisioni in modo.

\section{Scopo del lavoro}

Giornalmente il dipartimento di Protezione Civile comunica i dati relativi alla pandemia di Covid 19, per una migliore consultazione di tali dati si è pensato di sviluppare un’applicazione in grado di trasformarli automaticamente in grafici per una più facile lettura e comprensione dell’andamento della pandemia.
Lo scopo del seguente elaborato è quello di descrivere la struttura dell'applicazione creata per la realizzazione di questo servizio posto online, attraverso l’elaborazione dei dati relativi all’emergenza di Coronavirus Covid-19 utilizzando dei grafici interattivi.

\section{La Business Intelligence}
\subsection{Definizione}
La Business Intelligence o BI consiste nella trasformazione dei dati grezzi in informazioni utili e significative.
La Business Intelligence si può definire anche come un insieme di metodologie che spaziano da tutti i processi necessari per la raccolta dei dati, alle architetture e alle tecnologie  utilizzate per implementare tali processi, che trasformano dati in informazioni utili.

\subsection{L'importanza dei dati}
Nelle prime settimane della pandemia, la conoscenza della malattia e del suo trattamento è stata generata dalla condivisione di osservazioni aneddotiche e piccole serie di casi.

\noindent Nonostante gli operatori sanitari utilizzano la tecnologia moderna per comunicare, mai prima d'ora l'incapacità di costruire sistemi solidi di condivisione dei dati per analisi quasi in tempo reale su larga scala nell'assistenza sanitaria è stata più ovvia.

\noindent Nell'era delle cartelle cliniche elettroniche, i dati fisiologici, di laboratorio, di imaging, del processo decisionale e del trattamento vengono aggiornati continuamente. Le inferenze tratte da questi dati possono supportare le indagini epidemiologiche e guidare i protocolli di trattamento quando i dati degli studi clinici non sono disponibili o insufficienti per la gestione di una situazione in così rapida evoluzione.

\noindent Con il diffondersi della pandemia, i dati si accumulano, ma tendono a restare isolati all'interno dei sistemi ospedalieri. Per la lotta contro COVID-19 potrebbe essere molto utile condividere i dati disponibili con istituti di ricerca ed aziende farmaceutiche.

\noindent Sfortunatamente, i dati COVID-19 a livello di paziente non sono disponibili pubblicamente. 
Non esiste un database di questo tipo tecnologia o precedenti. 

\noindent Nonostante la tecnologia, non esiste un database globale liberamente accessibile, pertanto i dato sul Covid-19 a livello di paziente non sono disponibili pubblicamente.




