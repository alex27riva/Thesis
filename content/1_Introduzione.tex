\mcchap{Introduzione}{cap:intro}

Viviamo in un mondo sempre più data-driven che tradotto significa “guidato dai dati”, un’espressione efficace per indicare che le decisioni vengono prese in base all’analisi dei dati e quindi su fatti oggettivi e non su interpretazioni personali.
Raccogliere, analizzare e comprendere i dati è utile in ogni campo, dalla scienza, all'ingegneria, alla tecnologia alla matematica e così via, per poter attuare delle decisioni strategiche.\\
I dati sono fondamentali nei processi produttivi, nella logistica, nelle risorse umane, anche per la divisione amministrativa e nella gestione finanziaria.\\
La quantità di dati prodotti ogni giorno cresce in modo esponenziale, quindi è importante saper gestire questa mole di dati nel modo più efficiente ed efficace possibile, utilizzando le nuove tecnologie.
La visualizzazione dei dati aiuta a spiegare i contenuti, organizzandoli in un modo più comprensibile e mettendo in evidenza tendenze e valori anomali.\\
Una visualizzazione efficace consente di esporre i contenuti eliminando dai dati il superfluo e portando in primo piano le informazioni utili.\\
Sono moltissimi i metodi di visualizzazione disponibili per presentare i dati in modo efficace e interessante.


\section{Scopo del lavoro}

Giornalmente il dipartimento di Protezione Civile comunica i numeri dei contagi relativi alla pandemia di Covid 19.
Per una migliore consultazione si è pensato di sviluppare un’applicazione in grado di rappresentare i dati in modalità grafica, per facilitare la lettura e la comprensione dell’andamento della pandemia.
Lo scopo del seguente elaborato è quello di illustrare la struttura dell'applicazione creata per la realizzazione di questo servizio accessibile online attraverso l’elaborazione dei dati relativi all’emergenza di Coronavirus Covid-19 utilizzando dei grafici interattivi.

\section{La Business Intelligence}
La Business Intelligence o BI consiste nella trasformazione dei dati grezzi in informazioni utili e significative al fine dell’utilizzo in processi decisionali strategici, in modo che questi siano supportati da informazioni coerenti, aggiornate e altamente precise.
 
La Business Intelligence si può definire anche come un insieme di metodologie processi e strumenti che utilizzano le architetture e le tecnologie per implementare tutti i processi necessari per la raccolta dei dati, che attraverso elaborazioni, analisi o aggregazioni, ne permettono la trasformazione in informazioni, la loro conservazione, e la presentazione in una forma semplice, flessibile ed efficace, tale da costituire un supporto alle decisioni strategiche, tattiche ed operative.
 
Il concetto è stato introdotto negli anni ‘60, quando si è iniziato a parlare di sistemi per aiutare i manager nei processi decisionali, ma è a partire dagli anni ‘80 che si è diffuso in modo massiccio.
 
In pratica, un'applicazione di BI è un software che permette di trasformare grandi quantità di dati strutturati e anche de-strutturati in informazioni strategiche per l'attività dell'azienda. Il compito della BI è infatti quello di fornire report, statistiche, indicatori, grafici aggiornati utili ai decisori per effettuare scelte più corrette al fine della promozione del business e rendere, quindi, più competitive le realtà in cui operano.

\section{L'importanza dei dati}
Nelle prime settimane della pandemia, il Covid-19 ha rapidamente conquistato le prime pagine di giornali, telegiornali e siti web con informazioni scientifiche spesso prive di fondamento o provenienti da fonti poco attendibili.\\
Per far fronte a emergenze sanitarie di questo tipo è molto importante poter disporre di informazioni tempestive, comprensibili e il più possibile accurate per condividerle tra gli operatori sanitari.\\
I medici e gli operatori sanitari durante la pandemia hanno dovuto, oltre al loro consueto lavoro, raccogliere e fornire alle istituzioni una enorme quantità di informazioni impiegando la tecnologia moderna.\\
Però soltanto una minoranza degli ospedali italiani dispone di cartelle cliniche elettroniche, per poter raccogliere i dati in tempo reale.\\
Si è reso necessario costruire ad hoc dei sistemi informativi solidi per l’analisi e la condivisione dei dati su larga scala.
Questi sistemi informativi di Business Intelligence, di cui ne soffriamo la mancanza, sono doverosi per migliorare la conoscenza e la qualità della cura.\\
Nonostante la tecnologia, non esiste un database globale liberamente accessibile, pertanto i dati sul Covid-19 a livello di paziente non sono disponibili pubblicamente.





