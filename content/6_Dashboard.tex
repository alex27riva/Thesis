\mcchap{Le dashboard}{cap:dashboard}
\noindent In questo capitolo verranno presentate le schermate estratte dall'applicazione.
\section{Dashboard Italia}
Nella dashboard Italia vengono rappresentati i grafici che riassumono la situazione della pandemia in Italia.

\subsection{Nuovi casi}
Il grafico nella figura \ref{fig:nuovi_casi} mostra l'andamento giornaliero delle nuove persone che risultano positive al Covid-19.
\begin{figure}[htp]
    \centering
    \includegraphics[width=14cm]{nuovi_casi}
    \caption{Nuovi casi}
    \label{fig:nuovi_casi}
\end{figure}

\subsection{Totale casi}
Nella figura \ref{fig:totale_casi} sono rappresentati i casi totali delle persone che hanno contratto il virus, che comprende gli attuali positivi, i guariti e i decessi.
\begin{figure}[htp]
    \centering
    \includegraphics[width=14cm]{totale_casi}
    \caption{Totale casi}
    \label{fig:totale_casi}
\end{figure}

\subsection{Isolamento domiciliare}
Nella figura \ref{fig:isolamento_domiciliare} è raffigurato il grafico delle persone che sono in isolamento presso il loro domicilio.
\begin{figure}[htp]
    \centering
    \includegraphics[width=14cm]{isolamento_domiciliare}
    \caption{Isolamento domiciliare}
    \label{fig:isolamento_domiciliare}
\end{figure}

\subsection{Terapia intensiva}
Il grafico della figura \ref{fig:terapia_intensiva} mostra l'andamento dei pazienti ricoverati in ospedale in terapia intensiva.
\begin{figure}[htp]
    \centering
    \includegraphics[width=14cm]{terapia_intensiva}
    \caption{Terapia intensiva}
    \label{fig:terapia_intensiva}
\end{figure}

\subsection{Casi normalizzati}
Il grafico nella figura \ref{fig:casi_normalizzati} mostra il numero di casi normalizzati, ossia resi indipendente dai tamponi effettuati ogni giorno.
Si può anche osservare (in arancione) la linea della media mobile a 7 giorni.
\begin{figure}[htp]
    \centering
    \includegraphics[width=14cm]{casi_norm}
    \caption{Casi normalizzati}
    \label{fig:casi_normalizzati}
\end{figure}

\subsection{Terapia intensiva e ospedalizzati}
Nel grafico della figura \ref{fig:ti_ospdedalizzati} è possibile osservare l'andamento delle persone ospedalizzate, cioè la somma dei ricoverati con sintomi e della terapia intensiva.
\begin{figure}[htp]
    \centering
    \includegraphics[width=14cm]{ospedalizzati}
    \caption{Terapia intensiva e ospedalizzati}
    \label{fig:ti_ospdedalizzati}
\end{figure}

% inserire gli altri 2

\subsection{Nuovi casi e decessi}
Nel seguente grafico nella figura \ref{fig:casi_decessi} viene mostrato l'andamento dei positivi e dei decessi.
In questo grafico è anche possibile selezionare l'intervallo temporale, agendo sulla barra in basso.
\begin{figure}[htp]
    \centering
    \includegraphics[width=14cm]{casi_decessi}
    \caption{Nuovi casi e nuovi decessi}
    \label{fig:casi_decessi}
\end{figure}

\section{Dashboard Lombardia}
\subsection{Andamento dei contagi}
Nella figura \ref{fig:andamento_lomb} sono
\begin{figure}[htp]
    \centering
    \includegraphics[width=12cm]{img/lomb/andamento_lomb.png}
    \caption{Andamento dei contagi in Lombardia}
    \label{fig:andamento_lomb}
\end{figure}

\subsection{Percentuale positivi sui casi testati}
Nella figura \ref{fig:positivi_testati_lomb}
\begin{figure}[htp]
    \centering
    \includegraphics[width=12cm]{img/lomb/casi_tamp_lomb.png}
    \caption{Andamento dei contagi in Lombardia}
    \label{fig:positivi_testati_lomb}
\end{figure}

\subsection{Casi normalizzati}
Nellla figura \ref{fig:casi_norm_lomb} è possibile osservare
\begin{figure}[htp]
    \centering
    \includegraphics[width=12cm]{img/lomb/casi_norm_lomb.png}
    \caption{Andamento dei casi normalizzati}
    \label{fig:casi_norm_lomb}
\end{figure}

\subsection{Ospedalizzazioni}
Nellla figura \ref{fig:ospedalizzati_lomb} si può osservare che
\begin{figure}[htp]
    \centering
    \includegraphics[width=12cm]{img/lomb/ospedalizzati_lomb.png}
    \caption{Andamento delle ospedalizzazioni in Lombardia}
    \label{fig:ospedalizzati_lomb}
\end{figure}

\subsection{Decessi giornalieri}
Nellla figura \ref{fig:decessi_lomb} si può osservare che
\begin{figure}[htp]
    \centering
    \includegraphics[width=12cm]{img/lomb/decessi_lomb.png}
    \caption{Andamento dei deceduti}
    \label{fig:decessi_lomb}
\end{figure}

\subsection{Terapia intensiva}
Nellla figura \ref{fig:ti_lomb} si può osservare che
\begin{figure}[htp]
    \centering
    \includegraphics[width=12cm]{img/lomb/ti_lomb.png}
    \caption{Andamento della terapia intensiva}
    \label{fig:ti_lomb}
\end{figure}

\section{Dashboard Regioni}
La dashboard regioni permette di visualizzare i dati sulla pandemia per una regione a scelta.

\begin{figure}[htp]
    \centering
    \includegraphics[width=12cm]{dash_regioni}
    \caption{Visualizzazione della regione Abruzzo}
    \label{fig:dash_regioni}
\end{figure}